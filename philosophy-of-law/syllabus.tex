\documentclass[11pt,]{article}
\usepackage[margin=1in]{geometry}
\newcommand*{\authorfont}{\fontfamily{phv}\selectfont}
\usepackage[]{mathpazo}
\usepackage{abstract}
\renewcommand{\abstractname}{}    % clear the title
\renewcommand{\absnamepos}{empty} % originally center
\newcommand{\blankline}{\quad\pagebreak[2]}

\providecommand{\tightlist}{%
  \setlength{\itemsep}{0pt}\setlength{\parskip}{0pt}}
\usepackage{longtable,booktabs,tabu}

\usepackage{parskip}
\usepackage{titlesec}
\titlespacing\section{0pt}{12pt plus 4pt minus 2pt}{6pt plus 2pt minus 2pt}
\titlespacing\subsection{0pt}{12pt plus 4pt minus 2pt}{6pt plus 2pt minus 2pt}

\titleformat*{\subsubsection}{\normalsize\itshape}

\usepackage{titling}
\setlength{\droptitle}{-.25cm}

%\setlength{\parindent}{0pt}
%\setlength{\parskip}{6pt plus 2pt minus 1pt}
%\setlength{\emergencystretch}{3em}  % prevent overfull lines

\usepackage[T1]{fontenc}
\usepackage[utf8]{inputenc}

\usepackage{fancyhdr}
\pagestyle{fancy}
\usepackage{lastpage}
\renewcommand{\headrulewidth}{0.3pt}
\renewcommand{\footrulewidth}{0.0pt}
\lhead{}
\chead{}
\rhead{\footnotesize Phil 1700-02 and 03: The Examined Life: Ultimate
Questions (Philosophy) -- FALL 2023}
\lfoot{}
\cfoot{\small \thepage/\pageref*{LastPage}}
\rfoot{}

\fancypagestyle{firststyle}
{
\renewcommand{\headrulewidth}{0pt}%
   \fancyhf{}
   \fancyfoot[C]{\small \thepage/\pageref*{LastPage}}
}

%\def\labelitemi{--}
%\usepackage{enumitem}
%\setitemize[0]{leftmargin=25pt}
%\setenumerate[0]{leftmargin=25pt}

\newcommand{\pandocbounded}[1]{#1}


\makeatletter
\@ifpackageloaded{hyperref}{}{%
\ifxetex
  \usepackage[setpagesize=false, % page size defined by xetex
              unicode=false, % unicode breaks when used with xetex
              xetex]{hyperref}
\else
  \usepackage[unicode=true]{hyperref}
\fi
}
\@ifpackageloaded{color}{
    \PassOptionsToPackage{usenames,dvipsnames}{color}
}{%
    \usepackage[usenames,dvipsnames]{color}
}
\makeatother
\hypersetup{breaklinks=true,
            bookmarks=true,
            pdfauthor={ ()},
             pdfkeywords = {},
            pdftitle={Phil 1700-02 and 03: The Examined Life: Ultimate
Questions (Philosophy)},
            colorlinks=true,
            citecolor=blue,
            urlcolor=blue,
            linkcolor=magenta,
            pdfborder={0 0 0}}
\urlstyle{same}  % don't use monospace font for urls


\setcounter{secnumdepth}{0}

\usepackage{longtable}

\usepackage{graphicx}
% We will generate all images so they have a width \maxwidth. This means
% that they will get their normal width if they fit onto the page, but
% are scaled down if they would overflow the margins.
\makeatletter
\def\maxwidth{\ifdim\Gin@nat@width>\linewidth\linewidth
\else\Gin@nat@width\fi}
\makeatother
\let\Oldincludegraphics\includegraphics
\renewcommand{\includegraphics}[1]{\Oldincludegraphics[width=\maxwidth]{#1}}



\usepackage{setspace}

\title{Phil 1700-02 and 03: The Examined Life: Ultimate Questions
(Philosophy)}
\author{Montaque Reynolds}
\date{FALL 2023}


\begin{document}

		\maketitle
	

		\thispagestyle{firststyle}

%	\thispagestyle{empty}


	\noindent \begin{tabular*}{\textwidth}{ @{\extracolsep{\fill}} lr @{\extracolsep{\fill}}}


E-mail: \texttt{\href{mailto:montaque.reynolds@slu.edu}{\nolinkurl{montaque.reynolds@slu.edu}}} & Web: \href{http://montaque-reynolds.com}{\tt montaque-reynolds.com}\\
Office Hours: Tuesday 11am --- 1pm  &  Class Hours: Asychronous\\
Office: Adorjan Hall Rm. 235  & Class Room: Virtual\\
	&  \\
	\hline
	\end{tabular*}

\vspace{2mm}



\section{Course Description}\label{course-description}

\textbf{PHIL 1700 - The Examined Life: Ultimate Questions}
\textbf{Credits(s): 3 Credits}

This course invites students to explore enduring philosophical questions
and to reflectively evaluate the various answers given them by thinkers
from a range of social, historical, and religious contexts. Students
will tackle ultimate questions in a range of philosophical domains,
including issues such as the nature of self and ultimate reality,
morality and human meaning, rationality and the pursuit of truth. The
aim of the course is to give students an opportunity to critically
examine their own beliefs and commitments in dialogue with each other
and with great thinkers past and present. (Offered Fall, Spring, and
Summer)

\section{Course Learning Outcomes}\label{course-learning-outcomes}

The purpose of this course is to provide an introduction to some of the
most historically important philosophical texts, ideas, and thinkers as
well as to the distinctive activity of philosophy itself. Over the
course of the semester, students will:

• What is law? • Do laws have moral content? • What is the proper role
of judges in interpreting the law? • Is there an obligation to obey the
law? • How can rules give us reasons? • What, if anything, justifies
punishment by the state? • When are we morally responsible for what we
do? • What rights should/do our laws protect? • When, if ever, is
paternalistic interference by the state into the lives of its citizens
justified? • What special moral problems do lawyers face?

\section{Communications}\label{communications}

\subsection{Office Hours}\label{office-hours}

You do not need an appointment to come in during office hours. Please
just drop on by to ask questions, clarify lecture or readings, or just
to chat. In addition to the posted office hours, you can make an
appointment with me. (Just email me or talk with me before or after
class to set it up.) Outside of office hours, you can always stop by my
office, but while I will try to be there and available, I can't
guarantee that I will be able to meet with you. It never hurts to try,
but I may ask to schedule a meeting at another time. You can of course
always contact me by email.

\subsection{Official Communications}\label{official-communications}

I will from time to time wish to communicate with you outside of class
hours with important class related information. I will use your
school-supplied email. Because these communications are important, I
expect that you will check (and read!) your email at least once a day,
Monday through Friday.

\subsection{Email}\label{email}

The best way to reach me is by email. I will do my best to respond to
emails in a timely manner---usually within 24 hours---but I will
probably not answer emails between 5pm and 9am and on weekends. I am
much more likely to answer emails during weekdays. Please remember that
email communications are a type of formal communication. You should
always feel free to email me, but you should not treat an email to me
like texting. As is the case with any formal communication, your email
should contain a greeting---like ``Dear Professor Jacobs,'' or
``Jon,''---a fully explained description of what you are trying to
inform me of or ask me about, an ending with at least your first and
last name, and an informative subject line beginning with the Course,
number and semester, e.g., PHIL 1700-02 Spring. I'm not asking you to be
overly formal, but simply to observe some commonly expected norms of
communication. As you no doubt already know, this is how formal
communication---in academics and in ``the real world''---works.

\section{Required Readings}\label{required-readings}

In addition to materials that will be posted on canvas, the course
textbook is listed below:

list()

\section{READING SCHEDULE TOPICS
READINGS}\label{reading-schedule-topics-readings}

\subsection{1 Introduction}\label{introduction}

\subsubsection{I: What is Law?}\label{i-what-is-law}

\subsubsection{2--3 What is the difference between laws and commands?
Week 01, 08/15 -
08/19}\label{what-is-the-difference-between-laws-and-commands-week-01-0815---0819}

{[}Schauer{]} Austin, John. Selections from Lectures I, V, and VI of

``The Province of Jurisprudence Determined and the Uses of the Study of
Jurisprudence.''

See the Reading Notes

{[}Feinberg{]} Hart, H. L. A. ``Law as the Union of Primary and
Secondary Rules.''

Also available in:

---------. The Concept of Law. 2nd ed.~Edited by Penelope Bulloch and
Joseph Raz. Oxford University Press, 1997. ISBN: 9780198761235.

See the Reading Notes

\subsubsection{4 How do we determine the content of laws? Do laws have
moral content? Week 02, 08/22 -
08/26}\label{how-do-we-determine-the-content-of-laws-do-laws-have-moral-content-week-02-0822---0826}

{[}Schauer{]} Dworkin, Ronald. Selections from Chapters 2 and 4 of
``Taking Rights Seriously.''

See the Reading Notes

\subsubsection{5--6 How should judges interpret the law? Week 03, 08/29
- 09/02}\label{how-should-judges-interpret-the-law-week-03-0829---0902}

{[}Schauer{]} Dworkin, Ronald. Selections from Chapters 7 and 11 of
``Law's Empire.''

See the Reading Notes

{[}Feinberg{]} Scalia, Antonin. ``Common-Law Courts in a Civil-Law
System: The Role of United States Federal Courts in Interpreting the
Constitution and Laws.''

See the Reading Notes

{[}Feinberg{]} Dworkin, Ronald. ``Comment.''

See the Reading Notes

\subsection{II: The Obligation to Obey the
Law}\label{ii-the-obligation-to-obey-the-law}

\subsubsection{7--8 Is there an obligation to obey the law? Skepticism
about political obligation\ldots{} Week 04, 09/05 -
09/09}\label{is-there-an-obligation-to-obey-the-law-skepticism-about-political-obligation-week-04-0905---0909}

Wolff, Robert Paul. ``The Conflict Between Authority and Autonomy.'' In
In Defense of Anarchism. University of California Press, 1998. ISBN:
9780520215733.

See the Reading Notes

Smith, M. B. E. ``Is There a Prima Facie Obligation to Obey the Law?''
The Yale Law Journal 82, no. 5 (1973): 950--76.

See the Reading Notes

\subsubsection{9--10 Optimism about political obligation\ldots{} Week
05, 09/12 -
09/16}\label{optimism-about-political-obligation-week-05-0912---0916}

Dworkin, Ronald. Selections from Law's Empire. Belknap Press of Harvard
University Press, 1986, pp.~190--215. ISBN: 9780674518360.

See the Reading Notes

Raz, Joseph. ``Authority and Justification.'' Philosophy and Public
Affairs 14, no. 1 (1985): 3--29.

See the Reading Notes

\subsubsection{11 Should we sometimes disobey the law? Week 06, 09/19 -
09/23}\label{should-we-sometimes-disobey-the-law-week-06-0919---0923}

{[}Schauer{]} Rawls, John. ``The Justification of Civil Disobedience.''

See the Reading Notes

Optional Reading

{[}Feinberg{]} King, Martin Luther, Jr.~``Letter from Birmingham Jail.''

\subsection{III: Responsibility and
Punishment}\label{iii-responsibility-and-punishment}

\subsubsection{12 When exactly does an act cause harm? Week 07, 09/26 -
09/30}\label{when-exactly-does-an-act-cause-harm-week-07-0926---0930}

{[}Schauer{]} Hart, H. L. A., and Tony Honore. Selections from
``Causation In The Law''

See the Reading Notes

\subsubsection{13 Where does the burden of proof lie? Week 08, 10/03 -
10/07}\label{where-does-the-burden-of-proof-lie-week-08-1003---1007}

{[}Schauer{]} Kaplan, John. ``Decision Theory and the Factfinding
Process.''

See the Reading Notes

\subsubsection{14 What counts as evidence of guilt / liability? Week 09,
10/10 -
10/14}\label{what-counts-as-evidence-of-guilt-liability-week-09-1010---1014}

{[}Schauer{]} Thomson, Judith Jarvis. ``Liability and Individualized
Evidence.''

See the Reading Notes

\subsubsection{15--16 Is it appropriate to punish acts that
``successfully'' cause harm (e.g., murder, vehicular manslaughter) more
severely than similar acts which, merely as a matter of good luck, do
not? Week 10, 10/17 -
10/21}\label{is-it-appropriate-to-punish-acts-that-successfully-cause-harm-e.g.-murder-vehicular-manslaughter-more-severely-than-similar-acts-which-merely-as-a-matter-of-good-luck-do-not-week-10-1017---1021}

Cushman, Fiery. ``Crime and Punishment:

Distinguishing the Roles of Causal and Intentional Analyses in Moral
Judgment.'' Cognition 108, no. 2 (2008): 353--80.

See the Reading Notes

Lewis, David. ``The Punishment That Leaves Something to Chance.''
Philosophy and Public Affairs 18, no. 1 (1989): 53--67.

See the Reading Notes

\subsubsection{17--18 What, if anything, justifies punishment of
offenders by the state? Week 11, 10/24 -
10/28}\label{what-if-anything-justifies-punishment-of-offenders-by-the-state-week-11-1024---1028}

{[}Schauer{]} Bentham, Jeremy. Chapter 1 (Sections 1--11), and Chapters
13--14 from ``An Introduction to The Principles of Morals and
Legislation.''

See the Reading Notes

Rawls, John. ``Two Concepts of Rules.'' The Philosophical Review 64, no.
1 (1955): 3--32.

See the Reading Notes

\subsection{IV: Harm, Liberties, and the
Law}\label{iv-harm-liberties-and-the-law}

\subsubsection{19--20 When is the state justified in interfering in the
lives of its citizens? Week 12, 10/31 -
11/04}\label{when-is-the-state-justified-in-interfering-in-the-lives-of-its-citizens-week-12-1031---1104}

{[}Feinberg{]} Mill, John Stuart. ``The Liberal Argument from On
Liberty.'' Excerpts from Chapters 1 and 2, and Chapter IV.

See the Reading Notes

{[}Feinberg{]} Dworkin, Gerald. ``Paternalism.''

See the Reading Notes

\subsubsection{21--22 What are the justifications for and limits of
freedom of speech? Week 13, 11/07 -
11/11}\label{what-are-the-justifications-for-and-limits-of-freedom-of-speech-week-13-1107---1111}

Scanlon, Thomas. ``A Theory of Freedom of Expression.'' Philosophy and
Public Affairs 1, no. 2 (1972): 204--26.

See the Reading Notes

Langton, Rae. ``Speech Acts and Unspeakable Acts.'' Philosophy and
Public Affairs 22, no. 4 (1993): 293--330.

See the Reading Notes

\subsubsection{23 Can we harm people by bringing them into existence,
and should we be liable for that harm? Week 14, 11/14 -
11/18}\label{can-we-harm-people-by-bringing-them-into-existence-and-should-we-be-liable-for-that-harm-week-14-1114---1118}

Shiffrin, Seana Valentine.

``Wrongful Life, Procreative Responsibility, and the Significance of
Harm.'' (PDF) Legal Theory 5 (1999): 117--48.

See the Reading Notes

\subsection{V: Legal Ethics}\label{v-legal-ethics}

\subsubsection{24--26 What special ethical problems do lawyers face?
Week 15, 11/21 -
11/25}\label{what-special-ethical-problems-do-lawyers-face-week-15-1121---1125}

Wasserstrom, Richard. ``Lawyers as Professionals: Some Moral Issues.''
Human Rights Quarterly 5, no. 1 (1975): 105--28.

See the Reading Notes

Applbaum, Arthur. ``Professional Detachment: The Executioner of Paris.''
Harvard Law Review 109, no. 2 (1995): 458--86.

See the Reading Notes

\section{Grading and Evaluation}\label{grading-and-evaluation}

Because this is a shorter course, I have intended that each of the
assignments consecutively (structured) complement the other. The low
stakes small exercises below will give you the practice necessary to
complete the group project at the end. The argument development should
be a component part of the group project. Because the group project
consists of 3 parts with 2 students each part, I recommend that you
split each part into two and I will grade each component separately.

\subsection{Low-stakes Small Exercises 15\% of total
grade}\label{low-stakes-small-exercises-15-of-total-grade}

The small exercises should be no more than 500 words and are assigned
weekly, except for the end of semester. These are potentially worth 0-5
points for up to 15\% of your total grade. If you've done the assignment
it should be a pass (awarded the full score). See detailed accompanying
documentation for what these assignments look like and the assignment
rubric. The deadline is always Sunday 11:00 PM of the week of the
assignment. You need to submit these to a shared Canvas Module that will
be available to everyone who takes this course.

Each of the following assignments will be due the Wednesday at midnight
on the week specified for that assignment.

\subsection{Philosophical Music Criticism - Annotation Assignment
35\%}\label{philosophical-music-criticism---annotation-assignment-35}

Due: Week 03, 08/29 - 09/02

Sentiment analysis is a form of data analysis that focuses on extracting
meaningful content from language artifacts, e.g.~news paper articles,
novels, online reviews. Although sentiment typically refers to
extracting linguistic content that is emotionally meaningful, in this
project, we will focus on extracting philosophical content from public
documents that you have selected. In this case, these will be popular
entertainment media, namely music lyrics. There are two parts to this
project. First, select a favorite musical performance artist. Choose 5
songs by that artist on genius.com. Reading through the lyrics you've
chosen, copy every clause you find that you think corresponds with a
particular philosophical theory, in epistemology, metaphysics, ethics,
etc. Paste these into a separate word document line by line. In the next
part, you will select one of your annotations and defend your annotation
choice. First, articulate the philosophical foundation which you think
that particular annotation corresponds with, or is implied. Then tell
the reader why you think this is the case. I will be looking for
quotations of one or our readings which discusses that philosophical
foundation. More instruction and examples will be provided in the
related documents on canvas.

\subsection{Develop your own philosophical piece of fiction, 30\% of
total
grade}\label{develop-your-own-philosophical-piece-of-fiction-30-of-total-grade}

Due: Week 05, 09/12 - 09/16

Choose a song lyric, poem, rom-com, or game narrative, anywhere between
1500 and 3000 words and re-write it in such a way to highlight its
philosophical content (or do a fan fiction and change it in such a way
that it motivates philosophical discussion). E.g., you might turn it
into a philosophical thought experiment (e.g., Plato's cave, the trolley
problem) in an evocative and creative way. Then retell it again, but use
a different point of view. Then retell it a third time, again using a
different point of view. For example, you could write the trolley
problem in the first person, limited third person, omniscient narrator.
Try to ask yourself as you do this: how does changing the viewpoint
change our appreciation of this thought experiment?

\subsection{Group Project: 20\% of total
grade}\label{group-project-20-of-total-grade}

You will select your group members by the second week of classes: Week
02, 08/22 - 08/26

Assignment Due: Week 08, 10/03 - 10/07

For your group project, choose a piece of fiction from one of your
members. If anyone is not comfortable sharing theirs, let me know (I
will let you write one as a group. This could be very tricky however)
The project will be comprised of 3 parts, 2 persons each part, and each
group will be comprised of 6 students. If groups cannot be split evenly
into groups of 6, then one group will receive special instructions for
its project. Each student will be graded on one part of the project,
making each group member's grade (potentially) different. Students will
separate into groups by the second week of the semester, Week 02, 08/22
- 08/26, and decide which student will be responsible for which portion
of the project. I will provide an opportunity for each each class member
to get to know one another and select your groups. If this proves
difficult, please contact me and we will figure out a solution.

\subsection{Grading Scale}\label{grading-scale}

\begin{longtable}[]{@{}llll@{}}
\toprule\noalign{}
\endhead
\bottomrule\noalign{}
\endlastfoot
A & (4.0) & 94--100 & \\
A- & (3.7) & 90--94 & \\
B+ & (3.3) & 87--90 & \\
B & (3.0) & 83--87 & \\
B- & (2.7) & 80--87 & \\
C+ & (2.3) & 77--80 & \\
C & (2.0) & 73--77 & \\
C- & (1.7) & 70--73 & \\
D & (1.0 & 60--70 & \\
F & (0.0) & 0--6 & \\
\end{longtable}

\section{Missing or late Work/Exams}\label{missing-or-late-workexams}

\subsection{Extensions}\label{extensions}

There may be moments wherein situations arise and you cannot fulfill a
particular commitment that you've made to yourself, such as completing a
given assignment by a given date. I understand that sometimes, despite
our best efforts, things don't go according to plan. The printer breaks,
you get a cold, the idea didn't work out and you had to go back to the
drawing board. As I've learned from other professors here at SLU, it is
acceptable to offer grace for such moments. If you need an extension for
any reason, I will automatically grant one for three days (72 hours).
You must email me by midnight the \emph{day before} it is due. If you
email before then and ask for the extension, I will grant it. You do not
need to specify the reason you are asking for the extension. If you need
an extension for more than three days, your situation is likely serious
enough that you should be in contact with the Office of the Dean or the
Student Health Center. In such cases, I will be happy to be in contact
with those offices to arrange a schedule for you to complete your work.
If an extension is granted but you do not turn in the paper after three
days, the penalty will be 10\% for each 24 hours after the deadline.

\subsection{Extra Credit}\label{extra-credit}

There will be no opportunities for ``extra credit'' to improve grades
that have already been earned. Bargaining for grades (``I need a B
because\ldots{}'') is not acceptable. The only ways to achieve the grade
that you need are to do well on the quizzes and discussion boards and to
write excellent essays and argument evaluations.

\subsection{Withdrawals and
Incompletes}\label{withdrawals-and-incompletes}

Withdrawals and incompletes will be given only in accordance with
University policy. In particular, you should note that an incomplete is
given only in extraordinary circumstances.

\subsection{Expectations for Discussion Boards and Group
Projects/Presentations}\label{expectations-for-discussion-boards-and-group-projectspresentations}

Discussion is vital to a successful class and learning experience. Your
choice to participate in the class will ultimately determine your
ability to succeed in this class. If you are being extremely disruptive
in class, you will be asked to leave. Often, discussions may be on
controversial topics. This does not license anyone to yell at another
student or be in any way rude or ill-willed toward another student (or
to the instructor). I reserve the right to ask anyone to leave at any
time for being rude or disruptive.

\section{University Policies}\label{university-policies}

\subsection{ELECTRONIC DEVICE POLICY}\label{electronic-device-policy}

This is an asynchronous. This means that you will be ultimately
responsible for your own time. This includes what readings you will do
each week and when you will do those readings. Although this course is
only an eight week course, because it offers the same number of credits
as a full sixteen week course, there will be a lot of work due each
week. Perhaps seemingly more than a standard sixteen week course.

\subsection{ACADEMIC INTEGRITY SYLLABUS
STATEMENT}\label{academic-integrity-syllabus-statement}

Academic integrity is honest, truthful and responsible conduct in
all~academic endeavors.~The mission of Saint Louis University is ``the
pursuit of truth for the greater glory of God and for the service of
humanity.'' Accordingly, all acts of falsehood~demean and compromise the
corporate endeavors of teaching, research, health care, and community
service through which SLU fulfills its mission. The University strives
to prepare students for lives of personal and professional integrity,
and therefore regards all breaches of~academic~integrity~as matters of
serious concern. The full University-level~Academic~Integrity~Policy can
be found on the Provost's Office website at:
\url{https://www.slu.edu/provost/policies/academic-and-course/policy_academic-integrity_6-26-2015.pdf}.~
~ Additionally, each SLU College, School, and Center has its own
academic integrity policies, available on their respective websites.

The University reserves the right to penalize any student whose academic
conduct at any time is, in its judgment, detrimental to the University.
Such conduct shall include cases of plagiarism, collusion, cheating,
giving or receiving or offering or soliciting information in
examinations, or the use of previously prepared material in examinations
or quizzes in which such material had been banned. Violations should be
reported to your course instructor, who will investigate and adjudicate
them according to the policy on academic honesty of the College of Arts
and Sciences. If the charges are found to be true, the student may be
liable for academic or disciplinary probation, suspension, or expulsion
by the University. Students should review the College of Arts and
Sciences policy at

\url{http://www.slu.edu/colleges/AS/languages/department/files/AcademicHonestyPolicy.pdf}

\subsection{STUDENT SUCCESS CENTER SYLLABUS
STATEMENT}\label{student-success-center-syllabus-statement}

In recognition that people learn in a variety of ways and that learning
is influenced by multiple factors (e.g., prior experience, study skills,
learning disability), resources to support student success are available
on campus. The Student Success Center assists students with academic and
career related services, and it is located in the Busch Student Center
(Suite, 331) and the School of Nursing (Suite, 114). Students can visit
the website listed below to learn more about:

• Course-level support (e.g., faculty member, departmental resources,
etc.) by asking one's instructor. • University-level support (e.g.,
tutoring services, university writing services, disability services,
academic coaching, career services, and/or facets of curriculum
panning).

\url{http://www.slu.edu/life-at-slu/student-success-center/index.php}

\subsection{ACCESSIBILITY AND DISABILITY RESOURCES ACADEMIC
ACCOMMODATIONS SYLLABUS
STATEMENT}\label{accessibility-and-disability-resources-academic-accommodations-syllabus-statement}

Students with a documented disability who wish to request academic
accommodations must formally register their disability with the
University. Once successfully registered, students also must notify
their course instructor that they wish to use their approved
accommodations in the course.

Please contact the Center for Accessibility and Disability Resources
(CADR) to schedule an appointment to discuss accommodation requests and
eligibility requirements. Most students on the St.~Louis campus will
contact CADR, located in the Student Success Center and available by
email at
\href{mailto:accessibility_disability@slu.edu}{\nolinkurl{accessibility\_disability@slu.edu}}
or by phone at~314.977.3484. Once approved, information about a
student's eligibility for academic accommodations will be shared with
course instructors by email from CADR and within the instructor's
official course roster. Students who do not have a documented disability
but who think they may have one also are encouraged to contact to CADR.
Confidentiality will be observed in all inquiries.

\subsection{TITLE IX SYLLABUS
STATEMENT}\label{title-ix-syllabus-statement}

Saint Louis University and its faculty are committed to supporting our
students and seeking an environment that is free of bias,
discrimination, and harassment. If you have encountered any form of
sexual harassment, including sexual assault, stalking, domestic or
dating violence, we encourage you to report this to the University. If
you speak with a faculty member about an incident that involves a Title
IX matter, that faculty member must notify SLU's Title IX Coordinator
and share the~basic~facts~of your experience. This is true even if you
ask the faculty member not to disclose the incident. The Title IX
Coordinator will then be available to assist you in understanding all of
your options and in connecting you with all possible resources on and
off campus.

Anna Kratky is the Title IX Coordinator at Saint Louis University
(DuBourg Hall, room 36;
\href{mailto:anna.kratky@slu.edu}{\nolinkurl{anna.kratky@slu.edu}};~314-977-3886).
If you wish to speak with a confidential source, you may contact the
counselors at the University Counseling Center at 314-977-TALK or make
an anonymous report through SLU's Integrity Hotline by calling
1-877-525-5669 or online at
\url{http://www.lighthouse-services.com/slu}. To view SLU's policies,
and for resources, please visit the following web addresses:
\url{https://www.slu.edu/about/safety/sexual-assault-resources/index.php}
and \url{https://www.slu.edu/general-counsel}.

IMPORTANT UPDATE: SLU's Title IX Policy (formerly called the Sexual
Misconduct Policy) has been significantly revised to adhere to a new
federal law governing Title IX that was released on May 6, 2020. Please
take a moment to review the new policy and information at the following
web address:
\url{https://www.slu.edu/about/safety/sexual-assault-resources/index.php}.
Please contact Anna Kratky, the Title IX Coordinator, with any questions
or concerns.

\subsection{CONTENT DISCLAIMER}\label{content-disclaimer}

In this course, students may be required to read text or view materials
that they may consider offensive. The ideas expressed in any given text
do not necessarily reflect the views of the instructor, the Philosophy
Department, or Saint Louis University. Course materials are selected for
their historical and/or cultural relevance, or as an example of
stylistic and/or rhetorical strategies and techniques. They are meant to
be examined in the context of intellectual inquiry of the sort
encountered at the university level.

\subsection{WRITING CENTER}\label{writing-center}

I encourage you to take advantage of the writing services in the Student
Success Center by obtaining feedback benefits for writers at all skill
levels. Trained writing consultants can help with any writing,
multimedia project, or oral presentation. During the one-on-one
consultations, you can work on everything from brainstorming and
developing ideas to crafting strong sentences and documenting sources.
These services do fill up, so please make an appointment! For more
information, or to make, change, or cancel an appointment, call 977-3484
or visit

\url{https://www.slu.edu/life-at-slu/student-success-center/academic-support/university-writing-services/index.php}

\subsection{BASIC NEEDS SECURITY}\label{basic-needs-security}

Students in personal or academic distress and/or who may be specifically
experiencing challenges such as securing food or difficulty navigating
campus resources, and who believe this may affect their performance in
the course, are encouraged to contact the Dean of Students Office
(\href{mailto:deanofstudents@slu.edu}{\nolinkurl{deanofstudents@slu.edu}}
or 314-977-9378) for support. Furthermore, please notify the instructor
if you are comfortable in doing so, as this will enable instructors to
assist you with finding the resources you may need.




\end{document}

\makeatletter
\def\@maketitle{%
  \newpage
%  \null
%  \vskip 2em%
%  \begin{center}%
  \let \footnote \thanks
    {\fontsize{18}{20}\selectfont\raggedright  \setlength{\parindent}{0pt} \@title \par}%
}
%\fi
\makeatother
