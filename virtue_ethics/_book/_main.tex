% Options for packages loaded elsewhere
\PassOptionsToPackage{unicode}{hyperref}
\PassOptionsToPackage{hyphens}{url}
%
\documentclass[
]{book}
\usepackage{amsmath,amssymb}
\usepackage{iftex}
\ifPDFTeX
  \usepackage[T1]{fontenc}
  \usepackage[utf8]{inputenc}
  \usepackage{textcomp} % provide euro and other symbols
\else % if luatex or xetex
  \usepackage{unicode-math} % this also loads fontspec
  \defaultfontfeatures{Scale=MatchLowercase}
  \defaultfontfeatures[\rmfamily]{Ligatures=TeX,Scale=1}
\fi
\usepackage{lmodern}
\ifPDFTeX\else
  % xetex/luatex font selection
\fi
% Use upquote if available, for straight quotes in verbatim environments
\IfFileExists{upquote.sty}{\usepackage{upquote}}{}
\IfFileExists{microtype.sty}{% use microtype if available
  \usepackage[]{microtype}
  \UseMicrotypeSet[protrusion]{basicmath} % disable protrusion for tt fonts
}{}
\makeatletter
\@ifundefined{KOMAClassName}{% if non-KOMA class
  \IfFileExists{parskip.sty}{%
    \usepackage{parskip}
  }{% else
    \setlength{\parindent}{0pt}
    \setlength{\parskip}{6pt plus 2pt minus 1pt}}
}{% if KOMA class
  \KOMAoptions{parskip=half}}
\makeatother
\usepackage{xcolor}
\usepackage{longtable,booktabs,array}
\usepackage{calc} % for calculating minipage widths
% Correct order of tables after \paragraph or \subparagraph
\usepackage{etoolbox}
\makeatletter
\patchcmd\longtable{\par}{\if@noskipsec\mbox{}\fi\par}{}{}
\makeatother
% Allow footnotes in longtable head/foot
\IfFileExists{footnotehyper.sty}{\usepackage{footnotehyper}}{\usepackage{footnote}}
\makesavenoteenv{longtable}
\usepackage{graphicx}
\makeatletter
\newsavebox\pandoc@box
\newcommand*\pandocbounded[1]{% scales image to fit in text height/width
  \sbox\pandoc@box{#1}%
  \Gscale@div\@tempa{\textheight}{\dimexpr\ht\pandoc@box+\dp\pandoc@box\relax}%
  \Gscale@div\@tempb{\linewidth}{\wd\pandoc@box}%
  \ifdim\@tempb\p@<\@tempa\p@\let\@tempa\@tempb\fi% select the smaller of both
  \ifdim\@tempa\p@<\p@\scalebox{\@tempa}{\usebox\pandoc@box}%
  \else\usebox{\pandoc@box}%
  \fi%
}
% Set default figure placement to htbp
\def\fps@figure{htbp}
\makeatother
\setlength{\emergencystretch}{3em} % prevent overfull lines
\providecommand{\tightlist}{%
  \setlength{\itemsep}{0pt}\setlength{\parskip}{0pt}}
\setcounter{secnumdepth}{5}
\usepackage{booktabs}
\usepackage[]{natbib}
\bibliographystyle{plainnat}
\usepackage{bookmark}
\IfFileExists{xurl.sty}{\usepackage{xurl}}{} % add URL line breaks if available
\urlstyle{same}
\hypersetup{
  pdftitle={A Minimal Book Example},
  pdfauthor={John Doe},
  hidelinks,
  pdfcreator={LaTeX via pandoc}}

\title{A Minimal Book Example}
\author{John Doe}
\date{2024-10-01}

\begin{document}
\maketitle

{
\setcounter{tocdepth}{1}
\tableofcontents
}
\chapter{bibliography: {[}book.bib, packages.bib{]}}\label{bibliography-book.bib-packages.bib}

Placeholder

\section{Usage}\label{usage}

\section{Render book}\label{render-book}

\section{Preview book}\label{preview-book}

\chapter{The Good Life}\label{the-good-life}

\section{Virtue as a human excellence}\label{virtue-as-a-human-excellence}

\begin{enumerate}
\def\labelenumi{\arabic{enumi}.}
\tightlist
\item
  The first concerns the role in the human good life of activities and relationships that are, in their nature, especially vulnerable to reversal.
\end{enumerate}

\begin{itemize}
\tightlist
\item
  friendship
\item
  love
\item
  political activity
\item
  attachments to property or possessions
\end{itemize}

What is the role of these items in a good life, if one can easily loose these because of chance?

\begin{enumerate}
\def\labelenumi{\arabic{enumi}.}
\setcounter{enumi}{1}
\tightlist
\item
  The relationship among these external goods
\end{enumerate}

\begin{itemize}
\tightlist
\item
  Do they exist harmoniously?
\item
  Can they impair goodness of an agent's life?
\item
  Can they generate conflicting requirements?

  \begin{itemize}
  \tightlist
  \item
    E.g., can love cause someone to betray a friendship?
  \end{itemize}
\end{itemize}

\begin{enumerate}
\def\labelenumi{\arabic{enumi}.}
\setcounter{enumi}{2}
\tightlist
\item
  Self-suffiency, what is the ethical value of our appetites, feelings, and emotions, passions and sexuality?
\end{enumerate}

\begin{itemize}
\tightlist
\item
  Does the value of Self-suffiency outweigh the value of these other \emph{irrational attachments}?
\item
  Do they have value even though they can disrupt our own Self-suffiency? E.g., in rational planning?
\end{itemize}

\section{Examples of Fragility and Ambition}\label{examples-of-fragility-and-ambition}

\subsection{Aeschylus and practical conflict}\label{aeschylus-and-practical-conflict}

\begin{itemize}
\tightlist
\item
  What can we learn for tragic poetry and literature?
\end{itemize}

\begin{quote}
But the tragedies also show us, and dwell upon, another more intractable sort of case --- one which has come to be called, as a result, the situation of `tragic conflict'. In such cases we see a wrong action commetted without any direct physical compulsion and in full knowledge of its nature, by a person whose ethical character or commitments would otherwise dispose him to reject the act.
\end{quote}

\subsection{Sophocles' Antigon: conflict, vision, and simplification}\label{sophocles-antigon-conflict-vision-and-simplification}

\begin{itemize}
\tightlist
\item
  In response to what is learned from tragedy, we can simplify our value commitments.
\end{itemize}

\begin{quote}
For the claim is that the human being's relation to value in the world is not, or should not be, profoundly tragic: that it is, or should be, possible without culpable neglect or serious loss to cut off the risk of the typical tragic occurrence. Tragedy would then represent a primitive or benighted stage of ethical life and thought. {[}51{]}
\end{quote}

\subsection{Conclusion to Part I}\label{conclusion-to-part-i}

What have we learned?

\begin{itemize}
\tightlist
\item
  Values taken in the singular are vulnerable
\item
  \emph{Irrational attachments} can disrupt.
\item
  \emph{Irrational attachments} can become grounds of conflict.
\end{itemize}

But this was an over-ambitious attempt to eliminate luck from human life.

\begin{itemize}
\tightlist
\item
  This shows the importance of human value, \emph{rational choice}. {[}*tuch\(\'{e}\){]}
\end{itemize}

\section{Plato: Goodness without fragility}\label{plato-goodness-without-fragility}

Two problems:

\begin{enumerate}
\def\labelenumi{\arabic{enumi}.}
\tightlist
\item
  Dialogue
\item
  Development
\end{enumerate}

Some approaches

\begin{itemize}
\tightlist
\item
  lack of response to positive role of vulnerable values in the goodlife
\item
  Plato's insuffienct critique of tragic literature
\end{itemize}

\subsection{\texorpdfstring{The \emph{Protagoras}: a science of practical reasoning}{The Protagoras: a science of practical reasoning}}\label{the-protagoras-a-science-of-practical-reasoning}

How to develop a \emph{tuche}

\begin{itemize}
\tightlist
\item
  social political techne \(\rightarrow\) technai.
\item
  Important: defeated threats from physical environment, but what about the social environment?
\end{itemize}

How does science save and transform us?

\begin{itemize}
\tightlist
\item
  how do we rank activity independent of the feelings they produce?
\item
  how do we deal with the vulnerability and instability of individual human pursuits?
\end{itemize}

\subsection{Interlude I: Plato's anti-tragic theater}\label{interlude-i-platos-anti-tragic-theater}

Two ways of dealing with the question about mitigating luck in the social environment:

\begin{itemize}
\tightlist
\item
  Tragic theater: but irrational attachments can disrupt rational choice
\item
  techne (science): but lack of response to positive role of vulnerability in human values
\end{itemize}

E.g.,

\begin{quote}
Here, as in the \emph{Protagoras}, Plato very deliberately creates a speech that will give the impression of not having been deliberately formed. It is not artless; but its art is one that claims to go straight to the truth-telling part of the soul. It is simple rather than flowery, flat rather than emotive or persuasive. {[}132{]}
\end{quote}

\section{\texorpdfstring{The \emph{Republic}: true value and the standpoint of perfection}{The Republic: true value and the standpoint of perfection}}\label{the-republic-true-value-and-the-standpoint-of-perfection}

Defends a life of goodness without vulnerability.

\chapter{Cross-references}\label{cross}

Placeholder

\section{Chapters and sub-chapters}\label{chapters-and-sub-chapters}

\section{Captioned figures and tables}\label{captioned-figures-and-tables}

\chapter{Parts}\label{parts}

You can add parts to organize one or more book chapters together. Parts can be inserted at the top of an .Rmd file, before the first-level chapter heading in that same file.

Add a numbered part: \texttt{\#\ (PART)\ Act\ one\ \{-\}} (followed by \texttt{\#\ A\ chapter})

Add an unnumbered part: \texttt{\#\ (PART\textbackslash{}*)\ Act\ one\ \{-\}} (followed by \texttt{\#\ A\ chapter})

Add an appendix as a special kind of un-numbered part: \texttt{\#\ (APPENDIX)\ Other\ stuff\ \{-\}} (followed by \texttt{\#\ A\ chapter}). Chapters in an appendix are prepended with letters instead of numbers.

\chapter{Footnotes and citations}\label{footnotes-and-citations}

\section{Footnotes}\label{footnotes}

Footnotes are put inside the square brackets after a caret \texttt{\^{}{[}{]}}. Like this one \footnote{This is a footnote.}.

\section{Citations}\label{citations}

Reference items in your bibliography file(s) using \texttt{@key}.

For example, we are using the \textbf{bookdown} package \citep{R-bookdown} (check out the last code chunk in index.Rmd to see how this citation key was added) in this sample book, which was built on top of R Markdown and \textbf{knitr} \citep{xie2015} (this citation was added manually in an external file book.bib).
Note that the \texttt{.bib} files need to be listed in the index.Rmd with the YAML \texttt{bibliography} key.

The RStudio Visual Markdown Editor can also make it easier to insert citations: \url{https://rstudio.github.io/visual-markdown-editing/\#/citations}

\chapter{Blocks}\label{blocks}

Placeholder

\section{Equations}\label{equations}

\section{Theorems and proofs}\label{theorems-and-proofs}

\section{Callout blocks}\label{callout-blocks}

\chapter{Sharing your book}\label{sharing-your-book}

Placeholder

\section{Publishing}\label{publishing}

\section{404 pages}\label{pages}

\section{Metadata for sharing}\label{metadata-for-sharing}

\end{document}
