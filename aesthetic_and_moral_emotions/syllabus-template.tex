\documentclass[11pt,]{article}
\usepackage[margin=1in]{geometry}
\newcommand*{\authorfont}{\fontfamily{phv}\selectfont}
\usepackage[]{mathpazo}
\usepackage{abstract}
\renewcommand{\abstractname}{}    % clear the title
\renewcommand{\absnamepos}{empty} % originally center
\newcommand{\blankline}{\quad\pagebreak[2]}

\providecommand{\tightlist}{%
  \setlength{\itemsep}{0pt}\setlength{\parskip}{0pt}}
\usepackage{longtable,booktabs,tabu}

\usepackage{parskip}
\usepackage{titlesec}
\titlespacing\section{0pt}{12pt plus 4pt minus 2pt}{6pt plus 2pt minus 2pt}
\titlespacing\subsection{0pt}{12pt plus 4pt minus 2pt}{6pt plus 2pt minus 2pt}

\titleformat*{\subsubsection}{\normalsize\itshape}

\usepackage{titling}
\setlength{\droptitle}{-.25cm}

%\setlength{\parindent}{0pt}
%\setlength{\parskip}{6pt plus 2pt minus 1pt}
%\setlength{\emergencystretch}{3em}  % prevent overfull lines

\usepackage[T1]{fontenc}
\usepackage[utf8]{inputenc}

\usepackage{fancyhdr}
\pagestyle{fancy}
\usepackage{lastpage}
\renewcommand{\headrulewidth}{0.3pt}
\renewcommand{\footrulewidth}{0.0pt}
\lhead{}
\chead{}
\rhead{\footnotesize Aesthetic and Moral Emotions -- FALL 2023}
\lfoot{}
\cfoot{\small \thepage/\pageref*{LastPage}}
\rfoot{}

\fancypagestyle{firststyle}
{
\renewcommand{\headrulewidth}{0pt}%
   \fancyhf{}
   \fancyfoot[C]{\small \thepage/\pageref*{LastPage}}
}

%\def\labelitemi{--}
%\usepackage{enumitem}
%\setitemize[0]{leftmargin=25pt}
%\setenumerate[0]{leftmargin=25pt}

\newcommand{\pandocbounded}[1]{#1}


\makeatletter
\@ifpackageloaded{hyperref}{}{%
\ifxetex
  \usepackage[setpagesize=false, % page size defined by xetex
              unicode=false, % unicode breaks when used with xetex
              xetex]{hyperref}
\else
  \usepackage[unicode=true]{hyperref}
\fi
}
\@ifpackageloaded{color}{
    \PassOptionsToPackage{usenames,dvipsnames}{color}
}{%
    \usepackage[usenames,dvipsnames]{color}
}
\makeatother
\hypersetup{breaklinks=true,
            bookmarks=true,
            pdfauthor={ ()},
             pdfkeywords = {},
            pdftitle={Aesthetic and Moral Emotions},
            colorlinks=true,
            citecolor=blue,
            urlcolor=blue,
            linkcolor=magenta,
            pdfborder={0 0 0}}
\urlstyle{same}  % don't use monospace font for urls


\setcounter{secnumdepth}{0}

\usepackage{longtable}

\usepackage{graphicx}
% We will generate all images so they have a width \maxwidth. This means
% that they will get their normal width if they fit onto the page, but
% are scaled down if they would overflow the margins.
\makeatletter
\def\maxwidth{\ifdim\Gin@nat@width>\linewidth\linewidth
\else\Gin@nat@width\fi}
\makeatother
\let\Oldincludegraphics\includegraphics
\renewcommand{\includegraphics}[1]{\Oldincludegraphics[width=\maxwidth]{#1}}



\usepackage{setspace}

\title{Aesthetic and Moral Emotions}
\author{Montaque Reynolds}
\date{FALL 2023}


\begin{document}

		\maketitle
	

		\thispagestyle{firststyle}

%	\thispagestyle{empty}


	\noindent \begin{tabular*}{\textwidth}{ @{\extracolsep{\fill}} lr @{\extracolsep{\fill}}}


E-mail: \texttt{\href{mailto:montaque.reynolds@slu.edu}{\nolinkurl{montaque.reynolds@slu.edu}}} & Web: \href{http://montaque-reynolds.com}{\tt montaque-reynolds.com}\\
Office Hours: Tuesday 11am --- 1pm  &  Class Hours: Asychronous\\
Office: Adorjan Hall Rm. 235  & Class Room: Virtual\\
	&  \\
	\hline
	\end{tabular*}

\vspace{2mm}



\section{Course Description}\label{course-description}

This course invites students to explore philosophical questions
regarding our emotions. There are a number of issues this course will
cover, including relationships between our emotional states and moral
values. Objects of our emotions, what our emotions are directed at
whether fiction, real life, or cognitive. Our emotions as evaluative
appraisals of internal or external objects, for instance remembering a
painful moment. Our emotions and well-being, and finally our emotions
and our relationships with other sentient objects.

\section{Course Learning Outcomes}\label{course-learning-outcomes}

The purpose of this course is to provide an introduction to some of the
most historically important philosophical texts, ideas, and thinkers as
well as to the distinctive activity of philosophy itself. Over the
course of the semester, students will:

• What is an emotion? • Do emotions have moral content? • What is the
proper role of emotion in cognition? • Do we have an obligation to feel
a certain way? When? • How can emotions give us reasons? • What, if
anything, justifies an emotion? • When are we morally responsible for
what we feel?

\section{Communications}\label{communications}

\subsection{Office Hours}\label{office-hours}

You do not need an appointment to come in during office hours. Please
just drop on by to ask questions, clarify lecture or readings, or just
to chat. In addition to the posted office hours, you can make an
appointment with me. (Just email me or talk with me before or after
class to set it up.) Outside of office hours, you can always stop by my
office, but while I will try to be there and available, I can't
guarantee that I will be able to meet with you. It never hurts to try,
but I may ask to schedule a meeting at another time. You can of course
always contact me by email.

\subsection{Official Communications}\label{official-communications}

I will from time to time wish to communicate with you outside of class
hours with important class related information. I will use your
school-supplied email. Because these communications are important, I
expect that you will check (and read!) your email at least once a day,
Monday through Friday.

\subsection{Email}\label{email}

The best way to reach me is by email. I will do my best to respond to
emails in a timely manner---usually within 24 hours---but I will
probably not answer emails between 5pm and 9am and on weekends. I am
much more likely to answer emails during weekdays. Please remember that
email communications are a type of formal communication. You should
always feel free to email me, but you should not treat an email to me
like texting. As is the case with any formal communication, your email
should contain a greeting---like ``Dear Professor Jacobs,'' or
``Jon,''---a fully explained description of what you are trying to
inform me of or ask me about, an ending with at least your first and
last name, and an informative subject line beginning with the Course,
number and semester, e.g., PHIL 1700-02 Spring. I'm not asking you to be
overly formal, but simply to observe some commonly expected norms of
communication. As you no doubt already know, this is how formal
communication---in academics and in ``the real world''---works.

\section{Required Readings}\label{required-readings}

In addition to materials that will be posted on canvas, the course
textbook is listed below:

\section{READING SCHEDULE TOPICS
READINGS}\label{reading-schedule-topics-readings}

\subsection{Sensibility}\label{sensibility}

\subsubsection{The Emotional Construction of Morals
(Prinz)}\label{the-emotional-construction-of-morals-prinz}

\begin{enumerate}
\def\labelenumi{\arabic{enumi}.}
\tightlist
\item
  Sentimentalism by Michael Slote
\item
  The Discernment of Perception, Nussbaum 1990
\end{enumerate}

\textbf{Metaphysical Emotionism:}

\begin{itemize}
\tightlist
\item
  Darwall et all., 1992
\item
  McDowell (1985)
\item
  Wiggins (1987)
\item
  D'Arms and Jacobson (2006)
\end{itemize}

\textbf{Epistemic Emotionism:}

\begin{itemize}
\tightlist
\item
  Gibbard 1990
\item
  Ayer 1952
\item
  Stevenson (1937)
\item
  McMillen and Austin 1971
\end{itemize}

\subsubsection{Apt Imaginings (Gilmore)}\label{apt-imaginings-gilmore}

\textbf{Moral Judgments and the Emotions:}

\begin{itemize}
\tightlist
\item
  Greene et al 2001
\item
  Lerner et al 1998
\item
  Tye 1995, 100. Intentionality
\item
  Sentimentalism about Moral Understanding
\end{itemize}

\textbf{Representative Appraisal Theories:}

\begin{itemize}
\tightlist
\item
  Arnold 1960
\item
  Lazarus 1984
\item
  Smith and Ellsworth 1985; Scherer, Schorr, and Johnstone 2001; Smith
  and Lazarus (1993); Schacter and Singer (1962)
\item
  Greenspan (1988); de Sausa (1987); Roberts (1988)
\end{itemize}

\textbf{Cognitivists Theory of the Emotions:}

\begin{itemize}
\tightlist
\item
  Nussbaum 2001; Solomon 1993; Lyons 1980; Kenny 1963; Gordon 1990;
  Lazarus 1984; Lazarus 1991
\item
  Nussbaum 2004
\item
  Goldie 2000; Goldie 2009 (distinctive kind of evaluative state)
\item
  Prinz 2004
\end{itemize}

\begin{enumerate}
\def\labelenumi{\arabic{enumi}.}
\setcounter{enumi}{1}
\tightlist
\item
  Chapter 3 of Apt imaginings
\end{enumerate}

\subsection{Emotions and Well-Being}\label{emotions-and-well-being}

\subsubsection{Fragility of Goodness, Wandering in
Darkness}\label{fragility-of-goodness-wandering-in-darkness}

\textbf{Sensibility and Well-Being:}

\begin{itemize}
\tightlist
\item
  Nicomachean Ethics
\item
  Chapter 14 of Wandering in Darkness
\item
  Vulnerability of the Goodlife 1, and 2 (Fragility of Goodness)
\item
  Finely Aware and Richly Responsible, Nussbaum 1990
\end{itemize}

\textbf{Tragic Emotions:}

\begin{itemize}
\tightlist
\item
  FG Interlude 2: Luck and the tragic emotions
\item
  WD, Ch. 9, The Story of Job: Suffering and the Second-Personal
\item
  That Obscure Object of Desire: Pleasure in Painful Art {[}pdf{]}
\end{itemize}

\section{Grading and Evaluation}\label{grading-and-evaluation}

\subsection{Low-stakes Small Exercises 15\% of total
grade}\label{low-stakes-small-exercises-15-of-total-grade}

4 Sentence Outline

\subsection{Paper, Section 1, 35\%}\label{paper-section-1-35}

Due: Week 03, 08/29 - 09/02

\subsection{Paper Section 2, 30\% of total
grade}\label{paper-section-2-30-of-total-grade}

Due: Week 05, 09/12 - 09/16

\subsection{Paper, Section 3: 20\% of total
grade}\label{paper-section-3-20-of-total-grade}

You will select your group members by the second week of classes: Week
02, 08/22 - 08/26

Assignment Due: Week 08, 10/03 - 10/07

\subsection{Grading Scale}\label{grading-scale}

\begin{longtable}[]{@{}llll@{}}
\toprule\noalign{}
\endhead
\bottomrule\noalign{}
\endlastfoot
A & (4.0) & 94--100 & \\
A- & (3.7) & 90--94 & \\
B+ & (3.3) & 87--90 & \\
B & (3.0) & 83--87 & \\
B- & (2.7) & 80--87 & \\
C+ & (2.3) & 77--80 & \\
C & (2.0) & 73--77 & \\
C- & (1.7) & 70--73 & \\
D & (1.0 & 60--70 & \\
F & (0.0) & 0--6 & \\
\end{longtable}

\section{Missing or late Work/Exams}\label{missing-or-late-workexams}

\subsection{Extensions}\label{extensions}

There may be moments wherein situations arise and you cannot fulfill a
particular commitment that you've made to yourself, such as completing a
given assignment by a given date. I understand that sometimes, despite
our best efforts, things don't go according to plan. The printer breaks,
you get a cold, the idea didn't work out and you had to go back to the
drawing board. As I've learned from other professors here at SLU, it is
acceptable to offer grace for such moments. If you need an extension for
any reason, I will automatically grant one for three days (72 hours).
You must email me by midnight the \emph{day before} it is due. If you
email before then and ask for the extension, I will grant it. You do not
need to specify the reason you are asking for the extension. If you need
an extension for more than three days, your situation is likely serious
enough that you should be in contact with the Office of the Dean or the
Student Health Center. In such cases, I will be happy to be in contact
with those offices to arrange a schedule for you to complete your work.
If an extension is granted but you do not turn in the paper after three
days, the penalty will be 10\% for each 24 hours after the deadline.

\subsection{Extra Credit}\label{extra-credit}

There will be no opportunities for ``extra credit'' to improve grades
that have already been earned. Bargaining for grades (``I need a B
because\ldots{}'') is not acceptable. The only ways to achieve the grade
that you need are to do well on the quizzes and discussion boards and to
write excellent essays and argument evaluations.

\subsection{Withdrawals and
Incompletes}\label{withdrawals-and-incompletes}

Withdrawals and incompletes will be given only in accordance with
University policy. In particular, you should note that an incomplete is
given only in extraordinary circumstances.

\subsection{Expectations for Discussion Boards and Group
Projects/Presentations}\label{expectations-for-discussion-boards-and-group-projectspresentations}

Discussion is vital to a successful class and learning experience. Your
choice to participate in the class will ultimately determine your
ability to succeed in this class. If you are being extremely disruptive
in class, you will be asked to leave. Often, discussions may be on
controversial topics. This does not license anyone to yell at another
student or be in any way rude or ill-willed toward another student (or
to the instructor). I reserve the right to ask anyone to leave at any
time for being rude or disruptive.

\section{University Policies}\label{university-policies}

\subsection{ELECTRONIC DEVICE POLICY}\label{electronic-device-policy}

This is an asynchronous. This means that you will be ultimately
responsible for your own time. This includes what readings you will do
each week and when you will do those readings. Although this course is
only an eight week course, because it offers the same number of credits
as a full sixteen week course, there will be a lot of work due each
week. Perhaps seemingly more than a standard sixteen week course.

\subsection{ACADEMIC INTEGRITY SYLLABUS
STATEMENT}\label{academic-integrity-syllabus-statement}

Academic integrity is honest, truthful and responsible conduct in
all~academic endeavors.~The mission of Saint Louis University is ``the
pursuit of truth for the greater glory of God and for the service of
humanity.'' Accordingly, all acts of falsehood~demean and compromise the
corporate endeavors of teaching, research, health care, and community
service through which SLU fulfills its mission. The University strives
to prepare students for lives of personal and professional integrity,
and therefore regards all breaches of~academic~integrity~as matters of
serious concern. The full University-level~Academic~Integrity~Policy can
be found on the Provost's Office website at:
\url{https://www.slu.edu/provost/policies/academic-and-course/policy_academic-integrity_6-26-2015.pdf}.~
~ Additionally, each SLU College, School, and Center has its own
academic integrity policies, available on their respective websites.

The University reserves the right to penalize any student whose academic
conduct at any time is, in its judgment, detrimental to the University.
Such conduct shall include cases of plagiarism, collusion, cheating,
giving or receiving or offering or soliciting information in
examinations, or the use of previously prepared material in examinations
or quizzes in which such material had been banned. Violations should be
reported to your course instructor, who will investigate and adjudicate
them according to the policy on academic honesty of the College of Arts
and Sciences. If the charges are found to be true, the student may be
liable for academic or disciplinary probation, suspension, or expulsion
by the University. Students should review the College of Arts and
Sciences policy at

\url{http://www.slu.edu/colleges/AS/languages/department/files/AcademicHonestyPolicy.pdf}

\subsection{STUDENT SUCCESS CENTER SYLLABUS
STATEMENT}\label{student-success-center-syllabus-statement}

In recognition that people learn in a variety of ways and that learning
is influenced by multiple factors (e.g., prior experience, study skills,
learning disability), resources to support student success are available
on campus. The Student Success Center assists students with academic and
career related services, and it is located in the Busch Student Center
(Suite, 331) and the School of Nursing (Suite, 114). Students can visit
the website listed below to learn more about:

• Course-level support (e.g., faculty member, departmental resources,
etc.) by asking one's instructor. • University-level support (e.g.,
tutoring services, university writing services, disability services,
academic coaching, career services, and/or facets of curriculum
panning).

\url{http://www.slu.edu/life-at-slu/student-success-center/index.php}

\subsection{ACCESSIBILITY AND DISABILITY RESOURCES ACADEMIC
ACCOMMODATIONS SYLLABUS
STATEMENT}\label{accessibility-and-disability-resources-academic-accommodations-syllabus-statement}

Students with a documented disability who wish to request academic
accommodations must formally register their disability with the
University. Once successfully registered, students also must notify
their course instructor that they wish to use their approved
accommodations in the course.

Please contact the Center for Accessibility and Disability Resources
(CADR) to schedule an appointment to discuss accommodation requests and
eligibility requirements. Most students on the St.~Louis campus will
contact CADR, located in the Student Success Center and available by
email at
\href{mailto:accessibility_disability@slu.edu}{\nolinkurl{accessibility\_disability@slu.edu}}
or by phone at~314.977.3484. Once approved, information about a
student's eligibility for academic accommodations will be shared with
course instructors by email from CADR and within the instructor's
official course roster. Students who do not have a documented disability
but who think they may have one also are encouraged to contact to CADR.
Confidentiality will be observed in all inquiries.

\subsection{TITLE IX SYLLABUS
STATEMENT}\label{title-ix-syllabus-statement}

Saint Louis University and its faculty are committed to supporting our
students and seeking an environment that is free of bias,
discrimination, and harassment. If you have encountered any form of
sexual harassment, including sexual assault, stalking, domestic or
dating violence, we encourage you to report this to the University. If
you speak with a faculty member about an incident that involves a Title
IX matter, that faculty member must notify SLU's Title IX Coordinator
and share the~basic~facts~of your experience. This is true even if you
ask the faculty member not to disclose the incident. The Title IX
Coordinator will then be available to assist you in understanding all of
your options and in connecting you with all possible resources on and
off campus.

Anna Kratky is the Title IX Coordinator at Saint Louis University
(DuBourg Hall, room 36;
\href{mailto:anna.kratky@slu.edu}{\nolinkurl{anna.kratky@slu.edu}};~314-977-3886).
If you wish to speak with a confidential source, you may contact the
counselors at the University Counseling Center at 314-977-TALK or make
an anonymous report through SLU's Integrity Hotline by calling
1-877-525-5669 or online at
\url{http://www.lighthouse-services.com/slu}. To view SLU's policies,
and for resources, please visit the following web addresses:
\url{https://www.slu.edu/about/safety/sexual-assault-resources/index.php}
and \url{https://www.slu.edu/general-counsel}.

IMPORTANT UPDATE: SLU's Title IX Policy (formerly called the Sexual
Misconduct Policy) has been significantly revised to adhere to a new
federal law governing Title IX that was released on May 6, 2020. Please
take a moment to review the new policy and information at the following
web address:
\url{https://www.slu.edu/about/safety/sexual-assault-resources/index.php}.
Please contact Anna Kratky, the Title IX Coordinator, with any questions
or concerns.

\subsection{CONTENT DISCLAIMER}\label{content-disclaimer}

In this course, students may be required to read text or view materials
that they may consider offensive. The ideas expressed in any given text
do not necessarily reflect the views of the instructor, the Philosophy
Department, or Saint Louis University. Course materials are selected for
their historical and/or cultural relevance, or as an example of
stylistic and/or rhetorical strategies and techniques. They are meant to
be examined in the context of intellectual inquiry of the sort
encountered at the university level.

\subsection{WRITING CENTER}\label{writing-center}

I encourage you to take advantage of the writing services in the Student
Success Center by obtaining feedback benefits for writers at all skill
levels. Trained writing consultants can help with any writing,
multimedia project, or oral presentation. During the one-on-one
consultations, you can work on everything from brainstorming and
developing ideas to crafting strong sentences and documenting sources.
These services do fill up, so please make an appointment! For more
information, or to make, change, or cancel an appointment, call 977-3484
or visit

\url{https://www.slu.edu/life-at-slu/student-success-center/academic-support/university-writing-services/index.php}

\subsection{BASIC NEEDS SECURITY}\label{basic-needs-security}

Students in personal or academic distress and/or who may be specifically
experiencing challenges such as securing food or difficulty navigating
campus resources, and who believe this may affect their performance in
the course, are encouraged to contact the Dean of Students Office
(\href{mailto:deanofstudents@slu.edu}{\nolinkurl{deanofstudents@slu.edu}}
or 314-977-9378) for support. Furthermore, please notify the instructor
if you are comfortable in doing so, as this will enable instructors to
assist you with finding the resources you may need.




\end{document}

\makeatletter
\def\@maketitle{%
  \newpage
%  \null
%  \vskip 2em%
%  \begin{center}%
  \let \footnote \thanks
    {\fontsize{18}{20}\selectfont\raggedright  \setlength{\parindent}{0pt} \@title \par}%
}
%\fi
\makeatother
